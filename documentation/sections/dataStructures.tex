\section{Data Structures} \label{app:datastructure}

%% 1) Doubly Connected Edge List % http://www.ti.inf.ethz.ch/ew/lehre/CG12/lecture/Chapter%205.pdf


\paragraph{Curve}

%% class Circle:
%% DPE
%% IPE
%% firstDerivative
%% secondDerivative

%% class Curve:
%% def __init__(self, curve,
%% self.curve = curve
%% self.ipsIdx = itersectionPointsIdx
%% self.ipsTVal = tValue_at_intersections
%% self.ipsDer1st = at_intersections_derivative_1st
%% self.ipsDer2nd = at_intersections_derivative_2nd

%% assert len(self.ipsIdx) == len(self.ipsTVal)
%% assert len(self.ipsIdx) == len(self.ipsDer1st)
%% assert len(self.ipsIdx) == len(self.ipsDer2nd)


\paragraph{Node}

%% class Node:
%%     def __init__(self, point,
%%                  intersecting_curves_Idx=(),
%%                  intersecting_curves_Tval=()):
        
%%         self.point = point
%%         self.curveIdx = intersecting_curves_Idx
%%         self.curveTval = intersecting_curves_Tval


\paragraph{Half-Edge}

%% class HalfEdge:
%%     def __init__ (self,
%%                   selfIdx, twinIdx,
%%                   cIdx, side,
%%                   sIdx, sTVal, s1stDer, s2ndDer,
%%                   eIdx, eTVal, e1stDer, e2ndDer):

%%         self.selfIdx = selfIdx   # (sIdx, eIdx, pIdx)
%%         self.twinIdx = twinIdx   # twin half edge's index

%%         # half edge Curve's attributes:
%%         self.cIdx = cIdx         # Index of the curve creating the edge
%%         self.side = side         # defines the direction of t-value (in: t2>t1, out: t1>t2)
        
%%         # half edge attributes:
%%         self.sIdx = sIdx         # starting node's index        
%%         self.sTVal = sTVal
%%         self.s1stDer = s1stDer
%%         self.s2ndDer = s2ndDer

%%         self.eIdx = eIdx         # ending node's index
%%         self.eTVal = eTVal
%%         self.e1stDer = e1stDer
%%         self.e2ndDer = e2ndDer

\paragraph{Face}
%% (also, holes and patches)

%% class Face:
%%     def __init__(self, halfEdgeList, path):
%%         ''' '''
%%         self.halfEdges = halfEdgeList
%%         self.path = path
%%         self.holes = () # list of faces

%%     def get_area(self, considerHoles=True):
%%         '''
%%         Be aware that path.to_polygons() is an approximation of the face,
%%         if it contains curves, consequently the area would be approximated

%%         Green's theorem could provide an accurate measure of the area
%%         '''
%%         polygon = self.path.to_polygons()
%%         assert len(polygon) == 1
%%         x = polygon[0][:,0]
%%         y = polygon[0][:,1]
%%         PolyArea = 0.5*np.abs(np.dot(x,np.roll(y,1))-np.dot(y,np.roll(x,1)))
        
%%         holesArea = 0
%%         if considerHoles:
%%             for hole in self.holes:
%%                 polygon = hole.path.to_polygons()
%%                 assert len(polygon) == 1
%%                 x = polygon[0][:,0]
%%                 y = polygon[0][:,1]
%%                 holesArea += 0.5*np.abs(np.dot(x,np.roll(y,1))-np.dot(y,np.roll(x,1)))

%%         return PolyArea - holesArea

%%     def is_point_inside(self, point):
%%         ''' '''
%%         if self.path.contains_point( (point.x,point.y) ):
%%             for hole in self.holes:
%%                 if hole.path.contains_point( (point.x,point.y) ):
%%                     return False
%%             return True        
%%         return False

%%     def punch_hole(self, holeFace):
%%         ''' '''
%%         self.holes += (holeFace,)


%%     def get_punched_path(self):
%%         '''
%%         this is only useful for plotting
%%         the "path.contain_point()" doesn't work with holePunched pathes anyway

%%         no need to invert holes' trajectory,
%%         as they are supposedly superfaces, which means they have cw trajectory
%%         '''
%%         verts = self.path.vertices
%%         codes = self.path.codes

%%         for hole in self.holes:
%%             verts = np.append( verts, hole.path.vertices, axis=0)
%%             codes = np.append( codes, hole.path.codes)

%%         return mpath.Path(verts, codes)


        
\paragraph{Decomposition}
%% class Decomposition:
%%     def __init__ (self, graph, faces, superFaceIdx=None):
        
%%         self.graph = graph

%%         if superFaceIdx is not None:
%%             f = list(faces)
%%             self.superFace = f.pop(superFaceIdx)
%%             self.faces = tuple(f)
%%         else:
%%             self.superFace = None
%%             self.faces = faces

%%     def find_face(self, point):
%%         for idx,face in enumerate(self.faces):
%%             if face.is_point_inside(point):
%%                 return idx
%%         return None

%%     def find_neighbours(self, faceIdx):
%%         return []
        
%%     def get_extents(self):
%%         bboxes = [face.path.get_extents() for face in self.faces]
%%         return matplotlib.transforms.BboxBase.union(bboxes)


\paragraph{Subdivision} \quad
